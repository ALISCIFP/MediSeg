\section{Arbeitspakete}
\subsection{AP1: Input Pipeline}

Bauen eines Input Producers, welcher nebenläufig auf der CPU Daten einließt und
verschiedene Standardtechniken zur Data Augmention anwendet. Die Daten werden
bereits als Batch zusammengefasst und auf die Grafikkarte geladen. Verwaltet
wird das ganze von einer nebenläufigen Queue, welche neue Daten anfordert, wenn
die Anzahl der bereitgestellten Batches einen Schwellwert unterschreitet.

\subsection{AP2: Daten, Test- und Evaluierungsskripte}
Viele der folgenden Arbeitspakete beinhalten das Überprüfen verschiedener
Modelle auf zwei Kriterien:

\begin{itemize}
    \item \textbf{Geschwindigkeit}: Wie schnell wird ein einzelnes Bild
          segmentiert?
    \item \textbf{Qualität}: Wie brauchbar ist das Ergebnis?
\end{itemize}

Dabei wird in diesem Arbeitspaket ein Test-Datensatz festgelegt, welcher
konsistent für alle Modelle ist. Dieser wird niemals zum Training verwendet.
Die Qualität wird konsistent für alle Modelle berechnet werden. Das Maß der
Qualität ist noch festzulegen.


Die Geschwindigkeit ist die reine Zeit zum Segmentieren. Sie beinhaltet weder
die Zeit zum Laden des Modells noch zum schreiben des Ergebnisses auf die
Festplatte.

Es wird angegeben, wie gut die Qualität ist, wenn man einfach die häufigste Klasse schätzt. Damit wird ein Vergleich geschaffen, wie stark
die Verbesserung der verschiedenen Segmnetierer ist.


\subsection{AP3: Basic Segmenter}
Diese Modelle sollen eine Grundlage für die Beurteilung der Qualität und des
Wertes von weiteren Verbesserungen bieten.

\begin{itemize}
    \item Modell~301: Klassifiziere Pixel rein anhand der Farbe.
    \item Modell~302: Klassifiziere Pixel rein anhand der Farbe und Position.
    \item Modell~303: Nutze die Klassifikation der anderen Pixel um eine
          schlüssige Gesamtsegmentierung zu erhalten. Dafür können z.B.
          morphologische Operationen eingesetzt werden.
\end{itemize}

\clearpage
\subsection{AP4: Shallow CNN Segmenter}
\subsubsection{AP 4.1: SST-Modell}
Modell~401: Trainieren eines Segmentierers, welcher den Sliding-Window-Ansatz
benutzt und nur den Pixel im Zentrum klassifiziert. Es wird die folgende
Topologie haben:

\begin{table}[h!]
    \centering
    \begin{tabular}{lll}
    \toprule
    \textbf{Layer} & \textbf{Type} & \textbf{Shape} \\\midrule
    0     & Input       & 51\\
    1     & Convolution & 10 filter each $5\times 5$ \\
    2     & Convolution & 10 filter each $5\times 5$ \\
    3     & Pooling     & $2\times 2$ \\
    4     & Output      & 1 \\ \bottomrule
    \end{tabular}
\end{table}

Als Trainingsdaten werden $51 \times 51$-Patches mit einem Stride von $s=10$
aus den verfügbaren Bildern deterministisch ausgeschnitten.

Zur Klassifikation der Randpixel wird 0-Padding verwendet.

Diese Einstellungen wurden so gewählt, da sie bereits in~\cite{bittel2015pixel}
vernünftige Ergebnisse für eine semantische Segmentierungsaufgabe geliefert
haben.

\subsubsection{AP 4.2: Weitere Patch-Segmenter}
Trainieren von Segmentierern, welche den Sliding-Window-Ansatz benutzt und
mehr Pixel als nur das Zentrum klassifiziert. Variationen sind durch die
Window-Größe des Inputs und den Stride
(welcher damit die Ausgabe-Größe bestimmt) gegeben.

\clearpage
\subsection{AP6: FCN-Architectures}
Modell~6XX: Experimentiere mit FCN-Architekturen. Diese wurden
in~\cite{long2015fully} vorgestellt und haben für PASCAL VOC sehr gute
Segmentierungsergebnisse geliefert.


\subsection{AP7: Fine-Tuning}
Modell~7XX: Experimentiere mit Fine-Tuning Ansätzen. Aufgrund der geringen
Anzahl an verfügbaren Bildern zum trainieren könnte es sinnvoll sein ein
bereits auf Bildern trainiertes Netz zu nutzen und nur die letzten Schichten
des Netzwerks auf den medizinischen Daten anzupassen. In den ersten Schichten
werden typischerweise allgemeine Features trainiert, wohingegen die letzten
Schichten sehr problemspezifisch sind.


\subsection{AP8: Evaluation}
Vergleich der Ergebnisse und Schreiben des Abschluss-Papers. Das Paper wird
auf Englisch geschrieben und wird mindestens 3~Seiten lang sein. Es werden
die Ergebnisse der vorherigen Arbeitspakete vorgestellt, verglichen und
bewertet.