%!TEX root = Presentation-Thoma.tex
\section{Ende}
\subsection{Ausblick}
\begin{frame}{Ausblick}
    \begin{itemize}
        \item Einfluss von Data Augmentation
        \item Pixel+Umgebung mit Data Augmentation
        \item FCN: Peak Memory Usage und Zeit für die Segmentierung
        \item Alternativen zu VGG 16
        \item Conditional Random Fields (CRFs)
    \end{itemize}
\end{frame}

\subsection{Danke!}
\begin{frame}{Danke!}
    \begin{center}
        \Huge
	    Gibt es Fragen?
    \end{center}
\end{frame}

\subsection{Bildquellen}
\begin{frame}{Bildquellen}
\begin{itemize}
	\item VGG 16 architecture: D. Frossard. \textit{VGG in TensorFlow}. 2016.
    \item FCN: H. Noh, S. Hong, B. Han. \textit{Learning Deconvolution Network for Semantic Segmentation}. 2015.
\end{itemize}
\end{frame}

\subsection{Literatur}
\begin{frame}{Literatur}
\begin{itemize}
    \item M. Thoma: \textit{A Survey of Semantic Segmentation}. 2016. \href{http://arxiv.org/abs/1602.06541v2}{arxiv.org/abs/1602.06541v2}
    \item Jonathan Long, Evan Shelhamer, Trevor Darrell: \textit{Fully Convolutional Networks for Semantic Segmentation}. 2014. \href{https://arxiv.org/abs/1411.4038}{https://arxiv.org/abs/1411.4038}
\end{itemize}
\end{frame}

\subsection{Folien, \LaTeX und Material}
\begin{frame}{Folien, \LaTeX und Material}
Der Foliensatz sowie die \LaTeX und Ti\textit{k}Z-Quellen sind unter

\href{https://github.com/TensorVision/MediSeg/tree/master/AP8}{github.com/TensorVision/MediSeg/tree/master/AP8}
\\

Kurz-URL:
\href{http://tinyurl.com/medsim-segmentation}{tinyurl.com/medsim-segmentation}
\end{frame}
